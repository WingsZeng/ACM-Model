\documentclass{article}
\usepackage{xeCJK}
\usepackage{minted}
\usepackage{geometry}
\usepackage{graphicx}
% \usepackage{sourcecodepro}
\usepackage{setspace}
\usepackage{hyperref}

\setmainfont{Ubuntu}
\setsansfont{Ubuntu}
\setmonofont{Ubuntu Mono}
\setCJKmainfont{Noto Sans CJK SC}
\setCJKsansfont{Noto Sans CJK SC}
\setCJKmonofont{Noto Sans CJK SC}

\hypersetup{hidelinks,
	colorlinks=true,
	allcolors=black,
	pdfstartview=Fit,
	breaklinks=true
}

\newcommand{\cpp}[1]{\inputminted[linenos,breaklines,tabsize=4,mathescape]{c++}{#1}}
\geometry{a4paper,left=25mm,right=2cm,top=2cm,bottom=2cm}	% 调整页边距, 打印用
\usemintedstyle{bw}		% 设置minted的高亮风格

\begin{document}
\setcounter{section}{-1}	% section 从0开始标号
\title{\bf\Huge{常\ 用\ 算\ 法\ 模\ 板}}
\author{Wings}
\date{\today}

% 这里搞封面
\begin{titlepage}
\maketitle
\thispagestyle{empty}		% 封面无页数
\end{titlepage}
\newpage

% \linespread{1} \selectfont 	% 设置行距

% 目录
\setcounter{page}{1}
\pagenumbering{Roman}
\tableofcontents
\newpage

\setcounter{page}{1}
\pagenumbering{arabic}

\section{说明}

\subsection{约定}

\begin{itemize}
\setlength{\itemsep}{-5pt}	%设置item间距
\item \mintinline{c++}{const int MAXN} 为最大数据长度
\item \mintinline{c++}{const int MAXV} 为(抽象)图最大点数
\item \mintinline{c++}{const int MAXM} 为第二维数据最大长度或(抽象)图最大边数
\item \mintinline{c++}{const int MAXQ} 为最大询问
\item \mintinline{c++}{#ifdef GDB} 表示GDB调试, 从文件读取数据等
\item 若无特殊情况, 数组从下标1开始存储数据
\item 图论带边权一般以链式前向星建图, 且下标从 0 开始; 否则用向量存邻接点
\item[]
\item 为节省篇幅, 代码进行部分压行
\end{itemize}

\subsection{模板头}
\cpp{codes/head.cpp}

\newpage

\section{数据结构}

\subsection{并查集}

\subsubsection{基本操作}
\cpp{codes/data-struct/ufs.cpp}

\subsubsection{带边权}
\cpp{codes/data-struct/ufs-with-edge-weight.cpp}

\subsection{树状数组}
\cpp{codes/data-struct/bit.cpp}

\subsection{线段树}

\subsubsection{区间修改\&区间查询}
\cpp{codes/data-struct/segment-tree.cpp}

\subsubsection{动态开点\&合并}
\cpp{codes/data-struct/dynamic-segment-tree.cpp}

\subsection{ST表}
\cpp{codes/data-struct/st.cpp}

\subsection[主席树]{主席树(静态区间k小)}
\cpp{codes/data-struct/hjt-tree.cpp}

% \subsection{左偏树(可并堆)}
% \cpp{codes/data-struct/leftist-heap.cpp}

\subsection[FHQTreap]{FHQTreap(带区间翻转标记)}
\cpp{codes/data-struct/fhq-treap.cpp}

\subsection[Splay]{Splay (蒋叶桢指针版)}
\cpp{codes/data-struct/splay.cpp}

\section{动态规划}

\subsection{背包}

\subsubsection[01背包]{01背包\ 滚动数组 \& 输出方案}
\cpp{codes/dp/01-pack.cpp}

\subsubsection{完全背包}
\cpp{codes/dp/complete-pack.cpp}

\subsubsection{分组背包}
\cpp{codes/dp/group-pack.cpp}

\subsubsection[树形背包]{树形背包\ 以01树形背包为例}
\cpp{codes/dp/tree-pack.cpp}

% \subsubsection{多重背包}
% \cpp{codes/dp/multi-pack.cpp}

\subsection{最长公共子序列}
\cpp{codes/dp/lcm.cpp}

\subsection{优化}

% \subsubsection{四边形不等式}
% \cpp{codes/dp/quadrilateral.cpp}

% 我们要把转移方程化成$y = kx + b$的形式, 并且

% \begin{itemize}
% \setlength{\itemsep}{-5pt}	%设置item间距
% \item $y$ 是含有$dp[j]$的关于$j$的式子
% \item $kx$ 是同时含有$i, j$的式子$p\cdot f(i)g(j)$
% \item 其中 $k$ 为与$i$有关的元和其系数, $pf(i)$
% \item 而 $x$ 是$g(j)$
% \item $b$ 是含有$dp[i]$的关于$i$的式子
% \end{itemize}

% 根据"线性规划"思想,把点对$(x, y)$在坐标系中表示,然后将一条斜率为$k$的直线平移并且过一个点,使得$b$取得最值.那么这个使$b$取得最值的点即为最优决策,让$dp$从这里转移即可.

% x具有单调性的话, 可以用单调队列来维护凸包; k有单调性的话, 可以用双端队列来直接获得队头为最优决策点; k不具有单调性, 需要二分找最优决策点; x不具有单调性, 需要平衡树维护凸包;

% 只写过x和k都有单调性的, 抄个以前的代码.

% APIO特别行动队:给出长度为$n$的正整数数列$x$, 整数$a,b,c$, 把$x$分成若干段, 设该段的数和为$s$, 每一段的价值为$as^2 + bs + c$, 求最大价值.

% 设$s_i$为$x$的前缀和, 那么方程可以写成:
% $dp[i] = max\{dp[j] + a\cdot (s_i - s_j)^2 + b\cdot (s_i - s_j) + c\}$. 去掉$max$化简为$(dp[j] + as_j^2 - bs_j + c) = (2as_i)(s_j) + (dp[i] - as_i^2 - bs_i)$(括号是为了方便对应直线式)

% 最大化b, 维护下凸包. $x(s_j)$ 单调增, 单调队列维护下凸包. $k(2as_i)$单调减(注意$a<0$), 双端队列维护下凸包.

% \cpp{codes/dp/apio.cpp}

\subsubsection{决策单调性}
\cpp{codes/dp/decision-monotonicity.cpp}

\subsubsection{斜率优化}
\paragraph*{斜率单调, 横坐标单调}
\cpp{codes/dp/slope.cpp}

% \subsection*{插头DP}
% TODO

\section{树相关}

\subsection{点分治}
\cpp{codes/tree/centroid-decomposition.cpp}

\subsection{倍增LCA}
\cpp{codes/tree/double-lca.cpp}


% \subsection[虚树]{虚树(忘了, 抄了以前的代码)}
% TODO

% SDOI消耗战:给出一棵n个点的树, 带边权, m次询问, 每次问k个点, 把他们与1断开(割掉边)需要的最小权.
% \cpp{codes/tree/virtual-tree.cpp}

\section{图论}
\cpp{codes/graph/edge.cpp}

\subsection{Dijkstra}
\cpp{codes/graph/dijkstra.cpp}

\subsection[匈牙利]{匈牙利二分图匹配}
\cpp{codes/graph/hungarian.cpp}

\subsection{网络流}

\subsubsection{Dinic}
\cpp{codes/graph/network-flows/dinic.cpp}

\subsubsection[费用流]{费用流EK算法}
\cpp{codes/graph/network-flows/ek-mcmf.cpp}

\section{数学}

\subsection{取模函数}
\cpp{codes/math/mod.cpp}

\subsection{快速幂}
\cpp{codes/math/power.cpp}

\subsection{预处理组合数}
\cpp{codes/math/combination.cpp}

% \subsection{lucas}

\subsection{多项式}

\subsection{拉格朗日插值}

$$L(x) = \sum_{i=0}^{n-1} y_i \ell_i(x)$$
$$\ell_i(x) = \prod_{j=0,j\ne i}^{n-1} \frac{(x - x_j)}{(x_i - x_j)} = \frac{(x - x_0) \cdots (x - x_{i-1})(x - x_{i+1}) \cdots (x - x_{n-1})}{(x_i - x_0) \cdots (x_i - x_{i-1})(x_i - x_{i+1}) \cdots (x_i - x_{n-1})}$$
$$L(x) = \sum_{i=0}^{n-1} y_i \frac{\prod \limits_{j \ne i} (x - x_j)}{\prod \limits_{j \ne i} (x_i - x_j)}$$

\cpp{codes/math/polynomial/lagrange.cpp}

\subsubsection{横坐标连续拉格朗日插值}

$$pre_i(\xi) = \prod_{j=0}^i (\xi - x_j), suf_i(\xi) = \prod_{j=i}^{n-1} (\xi - x_j)$$
$$\ell_i(x) = \frac{pre_{i-1}(x) \cdot suf_{i+1}(x)}{(-1)^{n-1-i}(x_i - x_0)! (x_{n-1} - x_i)!}$$

\cpp{codes/math/polynomial/lagrange-consecutive.cpp}

\subsubsection{重心拉格朗日插值(增删点)}

$$w_i = \frac{1}{\prod_{j \ne i}(x_i - x_j)}$$
$$L(x) = \frac{\sum_{i=0}^{n-1}\frac{w_i}{x-x_i}y_i}{\sum_{i=0}^{n-1}\frac{w_i}{x-x_i}}$$

\cpp{codes/math/polynomial/lagrange-weight.cpp}

\subsection{FFT}

\cpp{codes/math/polynomial/fft.cpp}

\subsection{多项式乘法}

\cpp{codes/math/polynomial/polynomial-multiply.cpp}

% $$\begin{aligned}
% A(\omega_{len}^i) &= \sum_{k=0}^{\frac{n}{2}-1} a_{2k} (\omega_{len}^i)^{2k} + \sum_{k=0}^{\frac{n}{2}-1} a_{2k+1} (\omega_{len}^i)^{2k+1} \newline
% &= \sum_{k=0}^{\frac{n}{2}-1} a_{2k} (\omega_{len}^i)^{2k} + \omega_{len}^i\sum_{k=0}^{\frac{n}{2}-1} a_{2k+1} (\omega_{len}^i)^{2k} \newline
% &= \sum_{k=0}^{\frac{n}{2}-1} a_{2k} (\omega_{len}^{2i})^k + \omega_{len}^i\sum_{k=0}^{\frac{n}{2}-1} a_{2k+1} (\omega_{len}^{2i})^k \newline
% &= \sum_{k=0}^{\frac{n}{2}-1} a_{2k} (\omega_{\frac{len}{2}}^{i})^k + \omega_{len}^i\sum_{k=0}^{\frac{n}{2}-1} a_{2k+1} (\omega_{\frac{len}{2}}^{i})^k
% \end{aligned}$$

% $$\begin{aligned}
% A(\omega_{len}^{i+\frac{len}{2}}) &= \sum_{k=0}^{\frac{n}{2}-1} a_{2k} (\omega_{len}^{i+\frac{len}{2}})^{2k} + \sum_{k=0}^{\frac{n}{2}-1} a_{2k+1} (\omega_{len}^{i+\frac{len}{2}})^{2k+1} \newline
% &= \sum_{k=0}^{\frac{n}{2}-1} a_{2k} (\omega_{len}^{i+\frac{len}{2}})^{2k} + \omega_{len}^{i+\frac{len}{2}}\sum_{k=0}^{\frac{n}{2}-1} a_{2k+1} (\omega_{len}^{i+\frac{len}{2}})^{2k} \newline
% &= \sum_{k=0}^{\frac{n}{2}-1} a_{2k} (\omega_{len}^{2i+len})^k - \omega_{len}^i\sum_{k=0}^{\frac{n}{2}-1} a_{2k+1} (\omega_{len}^{2i+len})^k \newline
% &= \sum_{k=0}^{\frac{n}{2}-1} a_{2k} (\omega_{\frac{len}{2}}^{i})^k - \omega_{len}^i\sum_{k=0}^{\frac{n}{2}-1} a_{2k+1} (\omega_{\frac{len}{2}}^{i})^k
% \end{aligned}$$


\section{数论}

% \subsection{欧几里得}

\subsection{拓展欧几里得}
求 $ax+by = (a, b)$ 的一组解 $(x_0, y_0)$. 其他解满足: $x = x_0 + k \frac{b}{(a, b)}, y = y_0 - k \frac{a}{(a, b)}$
\cpp{codes/math/number-theory/exgcd.cpp}

\subsection{线性逆元}
\cpp{codes/math/number-theory/linear-inverse.cpp}

\subsection{欧拉函数}
\cpp{codes/math/number-theory/euler-phi.cpp}

\subsection{质因数分解}
\cpp{codes/math/number-theory/prime-factor-decomposition.cpp}

\subsection{线性筛素数}
\cpp{codes/math/number-theory/prime.cpp}

\subsection{数论分块}
\cpp{codes/math/number-theory/block-div.cpp}

\section{计算几何}

\subsection[误差修正]{误差修正(eps和sgn函数)}
\cpp{codes/computational-geometry/sgn.cpp}

\subsection{向量和点}
\cpp{codes/computational-geometry/vec.cpp}

\subsection{直线和线段}
\cpp{codes/computational-geometry/line.cpp}

\subsection{多边形}
\cpp{codes/computational-geometry/polygon.cpp}

\subsection{圆}
\cpp{codes/computational-geometry/circle.cpp}

\subsection{最小圆覆盖}
\cpp{codes/computational-geometry/min-circle-cover.cpp}

\subsection{平面最近点对}
\cpp{codes/computational-geometry/nearest-points.cpp}

\section{字符串}

\subsection{Trie}
\cpp{codes/string/trie.cpp}

\section{杂项}

\subsection{最长单调子序列}
\cpp{codes/misc/lis.cpp}

\subsection{莫队}

\subsubsection{普通莫队}
\cpp{codes/misc/mo.cpp}

\subsubsection{带修莫队}
\cpp{codes/misc/modifiable-mo.cpp}

\subsubsection{回滚莫队}
\cpp{codes/misc/rollback-mo.cpp}

\subsection{图论分块}
\cpp{codes/misc/graph-block.cpp}

\newpage

\section{附录}

\subsection{Vim配置}

新建文件 $\sim$/.vimrc

\inputminted[linenos,breaklines,tabsize=4,mathescape,texcl]{c}{codes/vimrc}

\subsection{Vim录制宏}

录制: 在 normal 模式下按q, 然后再按a-z中的某个字符(小写, 代表覆盖. 大写代表追加). 在normal模式下按q结束录制.

输入: 在 normal 模式下按下@, 再按下a-z中的某个字符, 输入对应字符保存的宏.

\subsection{GDB命令}

\begin{tabular}{ |l|l| }
	\hline
	\bfseries{功能}		& \bfseries{命令}		\\
	\hline
	设置断点				& b 行号/函数名			\\
	当条件满足时断点		& b 行号/函数名 if 条件	\\
	打印变量/表达式的值	& p 变量名/表达式			\\
	持续显示变量/表达式值	& disp 变量名/表达式		\\
	单步运行				& n						\\
	单步进入				& s						\\
	继续(下一个断点)		& c						\\
	退出					& q						\\
	(重新)运行			& r 					\\
	\hline
\end{tabular}

\subsection{对拍}

\subsubsection{Bash脚本}

数据生成器传入种子

\inputminted[linenos,breaklines,tabsize=4,mathescape,texcl]{bash}{codes/check-bash.sh}

\subsubsection{cmd批处理脚本}
\inputminted[linenos,breaklines,tabsize=4,mathescape,texcl]{bat}{codes/check-bat.bat}

\subsection{运行时检查}

运行时检查需要数据. 先生成极限数据, 再运行开了相应编译选项的程序, 输入极限数据.

\begin{spacing}{2}
\begin{center}
\begin{tabular}{ |c|c| }
	\hline
	检查未定义行为(数组越界, 爆int)	& -fsanitize=undefined		\\
	\hline
	检查(复杂)越界					& -fsanitize=address		\\
	\hline
\end{tabular}
\end{center}
\end{spacing}

\subsection{数学公式}

\subsubsection{几何}

\begin{spacing}{2}
\begin{center}
\begin{tabular}{ |c|c| }
	\hline
	缺球体积				& $V = \pi H^2\left(R-\frac{H}{3}\right)$		\\
	\hline
	两圆相交弓型高度/		& $h_1 = r_1 - \frac{r_1^2 - r_2^2 + d^2}{2d}$	\\
	两球相交缺球高度		& $h_2 = r_2 - \frac{r_2^2 - r_1^2 + d^2}{2d}$	\\
	\hline
\end{tabular}
\end{center}
\end{spacing}

\subsubsection{计数}

\begin{spacing}{2}
\begin{center}
\begin{tabular}{ |c|c| }
	\hline
	第二类斯特林数				& $S(n, k) = S(n-1, k-1) + k S(n-1, k), S(n, 0) = [n = 0]$		\\
	\hline
\end{tabular}
\end{center}
\end{spacing}

\end{document}
